\documentclass{article}
\usepackage[english]{babel}
\usepackage[utf8]{inputenc}
\usepackage{fancyhdr}
\usepackage{natbib}
\usepackage{graphicx}
\usepackage[]{algorithm2e}
\usepackage[procnames]{listings}
\usepackage{color}
\usepackage{tikz}
\usetikzlibrary{arrows}
\usetikzlibrary{calc,through,backgrounds}
\usepackage{pgfplots}
\pgfplotsset{compat=1.8}

\usepackage{tcolorbox}





\pagestyle{fancy}
\fancyhf{}

%% Header %%
\rhead{  JENA, RDF, PROTEGE & SPARQL| IKT437}
\lhead{Mandatory Exercise 1}
\rfoot{Page \thepage}


%% Front-page %%
\title{Jena project application\\  \large Mandatory exercise part 3}
\author{Christian Kråkevik, Per-Arne Andersen, Jahn Thomas Fidje \and Martin Hansen}
\date{November 2015}


\begin{document}

%% PYTHON HILIGHT %%
\definecolor{keywords}{RGB}{255,0,90}
\definecolor{comments}{RGB}{0,0,113}
\definecolor{red}{RGB}{160,0,0}
\definecolor{green}{RGB}{0,150,0}

\lstset{language=Java,
        basicstyle=\ttfamily\small,
        keywordstyle=\color{keywords},
        commentstyle=\color{comments},
        stringstyle=\color{red},
        showstringspaces=false,
        identifierstyle=\color{green},
        procnamekeys={def,class}}


\maketitle

\newpage

\section{Implementation}
The solution is a java application created with the java-jena library. The solution consist of several classes allowing the user
to create an ontology consisting of courses and topics with a custom vocabulary. \\


\noindent The solution consist of several different classes, a brief explanation of the following classes below :

\begin{description}
  \item[Main] The class responsible for running the program.
  \item[Topic] A class representing a single topic.
  \item[Course] Identical to the Topic class, only for topics.
  \item[UserInterface] UserInterface and logic.
  \item[SQuery] Queries the ontology with sparql.
  \item[BaseClasses] Spawns the ontology-classes (OntClass).
  \item[Triple] Adds triples to the OntModel.
  \item[VOC] The static vocabulary.
  \item[OProps] Serving static vars like NS.
  \item[ExportModel] Handles the saving of an OntModel to file (xml).
  \end{description}

  \vspace*{8px}


  \noindent The pages below describes the application in detail, it is sorted by program-flow.


\newpage

\subsection {Main}
The Main.class is only responsible for spawning the UserInterface, not a very interesting class.

\subsection {UserInterface}
The user interface spawns the UI and starts listening for user input. In the initialization it loads the ontology with some
starting / dummy-data. It also creates the vocabulary using the VOC class. This class also controls user-input and handles
the updating of the model.

\subsection {VOC}
The VOC or vocabulary class creates the static accessible ObjectProperties serving as the custom vocabulary for our ontology.
It also handles the creation of inverse properties. Once created the model is sent back to UserInterface and updated.

\subsection {BaseClasses}
The next step is to populate the ontology with some basic-example classes. These are created in this class. It creates the TOPIC, COURSE and LEARNINGTYPE
classes, these are used to spawn individuals later on. Once the classes are created we set the domain and ranges for the ObjectProperties in the VOC class.

\subsection {Triple}
The Triple class is the class responsible of submitting new triples to the model. This class takes in Individuals created from the Course and Topic class.
It then assigns the correct ObjectProperties and adds it as a triple format to the model. The model is then sent back and updated in the UserInterface class.

\subsection {SQuery}
The application also allows querying with SPARQL. The SQuery.class implements a wrapper, making it possible to ask it to query a given string
and return the result string. The SQuery is used in the raw-query tab (ui) and also as an internal query engine for the drop-down boxes (values) as
well as the actual querying for the questions / use cases in Part-3 of this mandatory exercise.


\newpage

\subsection {OProps}
OProps handles static settings which should be accessible trough the whole program. An example for this is the namespace, but also
the URI for the master-classes like TOPIC and COURSE. This makes the code more modular, and enables the user to change namespaces fast.
No variable value is hardcoded in the creation classes, (queries, triple-creation etc)!

\subsection {ExportModel}
The exportmodel handles exporting the Ontolgy to a file. The Ontology is exported in RDF/XML format. It is also written out in Turtle (ttl) format.

\subsection {Final notes}
The program is included in the hand-in. This .jar should be runnable for testing. A few demo-gifs is also included demonstrating the program.

\subsubsection {Changes since part 2}
- Rebuilt the code, splitting it into several classes, making it much more modular.
\newline- Rebuilt the ontology creation method. Removed hardcoding of variables.
\newline- Removed all ObjectProperty bugs, the ontology should now be correct.

\subsubsection {Missing functions}
- UC 3 is not implemented due to the design of the ontology. Only topics have requirements. (Our courses is more like a semester, and the topics is more like courses)





\vspace*{8px}


\end{document}
